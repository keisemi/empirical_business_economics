%% LyX 2.4.1 created this file.  For more info, see https://www.lyx.org/.
%% Do not edit unless you really know what you are doing.
\documentclass[11pt,japanese,nomag]{jsarticle}
\usepackage{textcomp}
\UseRawInputEncoding
\usepackage[letterpaper]{geometry}
\geometry{verbose,tmargin=1in,bmargin=1in,lmargin=1in,rmargin=1in}
\usepackage{color}
\usepackage{babel}
\usepackage{amsmath}
\usepackage{amsthm}
\usepackage{amssymb}
\usepackage{setspace}
\usepackage[authoryear]{natbib}
\onehalfspacing
\usepackage[bookmarks=true,bookmarksnumbered=true,bookmarksopen=false,
 breaklinks=false,pdfborder={0 0 1},backref=section,colorlinks=true]
 {hyperref}
\hypersetup{
 dvipdfmx,bookmarkstype=toc,urlcolor=black,linkcolor=blue,citecolor=red,linktocpage=false}

\makeatletter
%%%%%%%%%%%%%%%%%%%%%%%%%%%%%% User specified LaTeX commands.
%\usepackage{fullpage}
\usepackage{dcolumn}


%\usepackage{doublespace}


\usepackage{pxjahyper}

\makeatother

\begin{document}
\title{第7章付録: 2段階推定量の導出の詳細}
\date{最終更新: \today}
\maketitle
\begin{flushright}
\begin{small} 上武康亮・遠山祐太・若森直樹・渡辺安虎\\
「実証ビジネス・エコノミクス」 \\
 第7章「価格戦略をダイナミックに考える: \\
 単独エージェント動学7デルの推定[応用編]」\\
の付録\\
 \vspace{-0.2\baselineskip}
 \textcopyright 上武康亮・遠山祐太・若森直樹・渡辺安虎 \end{small} 
\par\end{flushright}

\section{はじめに}

この補足資料では、第7章で導入した2段階推定量に関する導出や各種定理について詳細な説明を行う。

\section{行列形式によるインバージョンの導出の詳細}

書籍7.4.1項「行列形式によるインバージョン」の導出の詳細について説明する。ノーテーションについては書籍を参照されたい。

まず、書籍 (7.2) 式の導出について取り上げよう。(7.2) 式の直前に示した事前の価値関数について、各状態変数で評価した値を以下のように書くことができる。
\begin{eqnarray*}
V(x_{1}) & = & \sum_{i}P(i\mid x_{1})\left\{ u(x_{1},d)+\beta[g(x_{1}\mid x_{1},d),\cdots,g(x_{|X|}\mid x_{1},d)]\left[\begin{array}{c}
V(x_{1})\\
\vdots\\
V(x_{|X|})
\end{array}\right]+\psi(x_{1},d)\right\} \\
 & \vdots\\
V(x_{|X|}) & = & \sum_{i}P(i\mid x_{|X|})\left\{ u(x_{|X|},d)+\beta[g(1\mid x_{|X|},d),\cdots,g(x_{|X|}\mid x_{|X|},d)]\left[\begin{array}{c}
V(x_{1})\\
\vdots\\
V(x_{|X|})
\end{array}\right]+\psi(x_{|X|},d)\right\} 
\end{eqnarray*}
この式を上からスタックする (重ね合わせる) ことで、以下の行列形式による (7.2) 式が得られるのである (ノーテーションは書籍参照)。
\[
V=\sum_{i}P(i)\odot\left[u(i)+\beta G(i)V+\psi(i)\right]
\]

続いて、本式を事前の価値関数のベクトル$V$について解いていこう。まず、 
\[
V-\beta\sum_{i}P(i)\odot\left[G(i)V\right]=\sum_{i}P(i)\odot\left[u(i)+\psi(i)\right]
\]
この式の左辺の第2項についてより細かく展開する。

\begin{align*}
\beta\sum_{i}P(i)\odot\left[G(i)V\right] & =\beta\sum_{i}\left(\begin{array}{c}
P(i\mid x_{1})\\
\vdots\\
P(i\mid x_{|X|})
\end{array}\right)\odot\left[\left(\begin{array}{ccc}
g(x_{1}\mid x_{1},d) & \cdots & g(x_{|X|}\mid x_{1},d)\\
\vdots & \ddots & \vdots\\
g(1\mid x_{|X|},d) & \cdots & g(x_{|X|}\mid x_{|X|},d)
\end{array}\right)\left(\begin{array}{c}
V(x_{1})\\
\vdots\\
V(x_{|X|})
\end{array}\right)\right]\\
 & =\beta\sum_{i}\left(\begin{array}{c}
P(i\mid x_{1})\\
\vdots\\
P(i\mid x_{|X|})
\end{array}\right)\odot\left(\begin{array}{c}
\sum_{x'}g(x'\mid x_{1},i)V(x')\\
\vdots\\
\sum_{x'}g(x'\mid x_{|X|},i)V(x')
\end{array}\right)\\
 & =\beta\sum_{i}\left(\begin{array}{ccc}
P(i\mid x_{1})g(x_{1}\mid x_{1},i) & \cdots & P(i\mid x_{1})g(x_{|X|}\mid x_{1},i)\\
\vdots & \ddots & \vdots\\
P(i\mid x_{|X|})g(x_{1}\mid x_{|X|},i) & \cdots & P(i\mid x_{|X|})g(x_{|X|}\mid x_{|X|},i)
\end{array}\right)\left(\begin{array}{c}
V(x_{1})\\
\vdots\\
V(x_{|X|})
\end{array}\right)\\
 & =\beta\sum_{i}\left[\left(P(i)\mathbf{1}_{(1\times|X|)}\right)\odot G(i)\right]V
\end{align*}
よって、 
\[
\left[I_{(|X|\times|X|)}-\beta\sum_{i}\left(P(i)\mathbf{1}_{(1\times|X|)}\right)\odot G(i)\right]V=\sum_{i}P(i)\odot\left[u(i)+\psi(i)\right]
\]
と書き下すことができ、この式について逆行列を取った計算を行うと、 
\[
V=\left[I_{(|X|\times|X|)}-\beta\sum_{i}\left(P(i)\mathbf{1}_{(1\times|X|)}\right)\odot G(i)\right]^{-1}\sum_{i}P(i)\odot\left[u(i)+\psi(i)\right]
\]
が得られる。

\section{Arcidiacono and Miller (2011) の補題の証明}

本節では、書籍7.4.2項「有限依存性アプローチ」で用いた 
\[
V(x)=\psi(x,i)+v(x,i),i=0,1
\]
について導出を行う。

事前の価値関数$V(x)$を以下のように展開する。 
\begin{align*}
V(x) & =\int\max_{d}\left\{ v(x,d)+\epsilon(d)\right\} dG(\epsilon)\\
 & =\int\max_{d}\left\{ v(x,d)-v(x,i)+\epsilon(d)\right\} dG(\epsilon)+v(x,i)
\end{align*}
ここで選択$i$は任意のもので構わない。一般のケースにおいては、

\[
\psi(x,i)=\int\max_{d}\left\{ v(x,d)-v(x,i)+\epsilon(d)\right\} dG(\epsilon)
\]
と定義することで、$V(x)=\psi(x,i)+v(x,i)$と書くことができる。

さて、各選択肢に特有なショック項$\epsilon(d)$が独立かつ同一な第1種極値分布に従うケースを考えよう。 このとき、$\psi(x,i)$はログサム公式を使うことで以下のように書き下すことができる。
\begin{align*}
\psi(x,i) & =\int\max_{d}\left\{ v(x,d)-v(x,i)+\epsilon(d)\right\} dG(\epsilon)\\
 & =\log\sum_{d}\exp\left(v(x,d)-v(x,i)\right)+\gamma
\end{align*}
ここで$\gamma$はオイラー定数である。

さらに、Hotz-Millerのインバージョンより、$v(x,d)-v(x,i)=\log P(d\mid x)-\log P(i\mid x)$と書くことができる。この式を代入すると、
\begin{align*}
\psi(x,i) & =\log\sum_{d}\exp\log\frac{P(d\mid x)}{P(i\mid x)}+\gamma\\
 & =\log\sum_{d}\frac{P(d\mid x)}{P(i\mid x)}+\gamma\\
 & =\log\frac{1}{P(i\mid x)}\sum_{d}P(d\mid x)+\gamma\\
 & =-\log(P(i\mid x))+\gamma
\end{align*}
と得られる。

\section{Agguiregabiria and Mira (2002) で用いた等式の証明}

本節では書籍7.4.1項「行列形式によるインバージョン」の導出で利用した 
\begin{align*}
\mathbb{E}[\epsilon(i)\mid i\ \text{is\ optimal},x] & =\gamma-\log P(i\mid x)
\end{align*}
という性質について導出を行う。

まず、 
\begin{align*}
\mathbb{E}[\epsilon(i)\mid i\ \text{is\ optimal},x] & =\mathbb{E}[\epsilon(i)\mid v(x,i)+\epsilon(i)\geq v(x,j)+\epsilon(j),\forall j]
\end{align*}
として定義される。これは、選択肢$i$において最も高い効用が得られることを条件づけしたときの$\epsilon(i)$に関する期待値である。

以下の導出では状態変数$x$については表記を割愛する。(状態変数$x$について条件づけした議論を考えてもらえれば良い。) また、選択肢として$i$と$j$という2つがある状況を考える。\footnote{より一般のケースの導出については、StackExchange 'Expectation of the Maximum of iid
Gumbel Variables' の記事を参照されたい。https://stats.stackexchange.com/questions/192424/expectation-of-the-maximum-of-iid-gumbel-variables} さらに、選択肢のノーテーションとして下付きの文字を使う。

まず、 
\begin{eqnarray}
\mathbb{E}[\epsilon_{i}\mid v_{i}+\epsilon_{i}\geq v_{j}+\epsilon_{j},\forall j\neq i] & = & \frac{1}{P(i)}\underbrace{\int\int\epsilon_{i}\mathbf{1}\left\{ v_{i}+\epsilon_{i}\geq v_{j}+\epsilon_{j},\forall j\neq i\right\} dG(\epsilon_{i})dG(\epsilon_{j})}_{\equiv A}\label{eq:fafa}
\end{eqnarray}
ここで、$G(\epsilon)$は第1種極値分布の分布関数である。後ほど使う密度関数$g(\epsilon)$と合わせて以下の記述する。
\begin{align*}
g(\epsilon) & =\exp(-\epsilon)\exp\left(-\exp(-\epsilon)\right)\\
G(\epsilon) & =\exp\left(-\exp(-\epsilon)\right)
\end{align*}
である。なお、この積分は選択肢特有のショック$\epsilon_{i},\epsilon_{j}$に関する多重積分であるが、これらは独立のショックであることに留意されたい。

上記の式の右辺の多重積分の項$A$についてさらに展開すると、 
\[
A=\int_{-\infty}^{\infty}\epsilon_{i}\left[\int_{-\infty}^{\infty}\mathbf{1}\left\{ v_{i}-v_{j}+\epsilon_{i}\geq\epsilon_{j}\right\} dG(\epsilon_{j})\right]dG(\epsilon_{i})
\]
となる。中の積分$\int_{-\infty}^{\infty}\mathbf{1}\left\{ v_{i}-v_{j}+\epsilon_{i}\geq\epsilon_{j}\right\} dG(\epsilon_{j})$は分布関数で評価することができ、
\begin{align*}
\int_{-\infty}^{\infty}\mathbf{1}\left\{ v_{i}-v_{j}+\epsilon_{i}\geq\epsilon_{j}\right\} dG(\epsilon_{j}) & =G\left(v_{i}-v_{j}+\epsilon_{i}\right)\\
 & =\exp\left(-\exp(-v_{i}+v_{j}-\epsilon_{i})\right)
\end{align*}
となる。よって、 
\begin{align*}
A & =\int_{-\infty}^{\infty}\epsilon_{i}\exp\left(-\exp(-v_{i}+v_{j}-\epsilon_{i})\right)\exp(-\epsilon_{i})\exp\left(-\exp(-\epsilon_{i})\right)d\epsilon_{i}\\
 & =\int_{-\infty}^{\infty}\epsilon_{i}\exp\left(-\left(1+\exp(-v_{i}+v_{j})\right)\exp(-\epsilon_{i})\right)\exp(-\epsilon_{i})d\epsilon_{i}
\end{align*}
この積分を計算するべく、いくつかのノーテーションの定義を行う。まず、$D=1+\exp(-v_{i}+v_{j})$とする。そのうえで、以下の変数$t$を用いた置換積分を行う。
\begin{align*}
t & =D\exp(-\epsilon_{i})
\end{align*}
ここで、 
\[
dt=-D\exp(-\epsilon_{i})d\epsilon_{i}
\]
である。また、積分の範囲について考えると、$\epsilon_{i}\rightarrow\infty$のとき、$t\rightarrow0$であり、$\epsilon_{i}\rightarrow-\infty$のとき、$t\rightarrow\infty$である。この点を踏まえて積分の計算を進めていくと
\[
A=\int_{\infty}^{0}\left(\log(D)-\log(t)\right)\exp\left(-t\right)\left(-\frac{1}{D}\right)dt
\]
となる。さらに展開すると、 
\begin{align*}
A & =\frac{1}{D}\left[\log(D)\int_{0}^{\infty}\exp(-t)dt-\int_{0}^{\infty}\left(\log(t)\exp(-t)\right)dt\right]\\
 & =\frac{\log D}{D}\left(\int_{0}^{\infty}\exp(-t)dt\right)-\frac{1}{D}\int_{0}^{\infty}\left(\log(t)\exp(-t)\right)dt
\end{align*}
ここで、$\int_{0}^{\infty}\exp(-t)dt=1$であり、また$-\int_{0}^{\infty}\left(\log(t)\exp(-t)\right)dt=\gamma$となりこれはオイラー定数となる。したがって、
\begin{align*}
A & =\frac{\log D+\gamma}{D}
\end{align*}
となる。

ここで、$D$について書き直すと、 
\begin{align*}
D & =1+\exp(-v_{i}+v_{j})\\
 & =\exp(-v_{i})\left(\exp(v_{i})+\exp(v_{j})\right)\\
 & =\frac{\exp(v_{i})+\exp(v_{j})}{\exp(v_{i})}\\
 & =\frac{1}{P(i)}
\end{align*}
となる。なお、今考えている設定では、選択肢$i,j$の2つであるため、 
\[
P(i)=\frac{exp(v_{i})}{exp(v_{i})+\exp(v_{j})}
\]
となることに留意されたい。

よって、元の (\ref{eq:fafa}) 式に戻ると、 
\begin{eqnarray*}
\mathbb{E}[\epsilon_{i}\mid v_{i}+\epsilon_{i}\geq v_{j}+\epsilon_{j},\forall j\neq i] & = & \frac{1}{P(i)}\frac{\log D+\gamma}{D}\\
 & = & \log D+\gamma\\
 & = & \gamma-\log(P(i))
\end{eqnarray*}
として目的のものが得られた。 
\end{document}
